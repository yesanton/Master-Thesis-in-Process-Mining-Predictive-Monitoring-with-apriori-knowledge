%!TEX root = ./BPM17.tex

\section{Conclusions} % (fold)
\label{sec:conclusions}

%In this paper we have presented a family of techniques presented in this paper, together with their careful evaluation, aims at addressing the insertion of apriori knowledge in predictive business process monitoring algorithms. Our contribution is twofold. In Section~\ref{sec:our_approach} have three techniques based on Recurrent Neural Networks (RNN) able to leverage knowledge on the dataset structure as well as apriori knowledge for predicting the sequence of future activities in ongoing traces, while in Section~\ref{sec:evaluation} we have shown that taking into account the dataset structure as well as the apriori knowledge increases the accuracy of the results over a plain RNN-based baseline.
In this paper, we have presented two techniques based on RNNs with LSTM cells able to leverage knowledge about the structure of the process execution traces as well as a-priori knowledge about their future development for predicting the sequence of future activities of an ongoing case. In particular, we show that opportunely tailoring LSTM-based algorithms it is possible to take into account a-priori knowledge at prediction time without the need to retrain the predictive algorithms in case new knowledge becomes available. The results of our experiments show that \nocycle correctly deals with the presence of cycles in the logs and \protrack is able to correctly leverage a-priori knowledge in a way that it performs better with logs characterized by a low degree of sparsity of activity labels and when the a-priori knowledge constrains the behavior of the process more.

Future work will include: (i) dealing with more complex forms of a-priori knowledge. In particular, we aim at addressing a-priori knowledge on activities and on their data payload, as well as dynamic knowledge that can evolve in the future of an ongoing case; (ii) extending the proposed algorithms to leverage a-priori knowledge also for other types of predictions; (iii) extending the experimental evaluation especially focusing on the investigation of metrics for evaluating the influence on the predictions of the different degrees of freedom/strength of the a-priori knowledge; and (iv) inserting the presented techniques in predictive business process monitoring frameworks such as the one discussed in~\cite{Di-Francescomarino:2016aa}.

% section conclusions (end) 