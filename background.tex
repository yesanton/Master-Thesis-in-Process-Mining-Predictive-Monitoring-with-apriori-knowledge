%!TEX root = ./BPM17.tex


\section{Background} % (fold)
\label{sec:background}
%%%% CONTENT
% background + related work
In this section, we report the background concepts useful for understanding the remainder of the thesis.



\subsection{Event Logs and Traces}

Most of the huge enterprises and middle sized companies have a lot of processes being executed. With the rapid development of database technology and inexpensiveness of data warehouses saving all the transactional data in the company became a must. Most of this data is about processes that are happening in the company. Those range from core processes such as "order to cash" and "issue to resolution" to additional such as managerial processes. The nature of this information is processlike (sequential) so they are usually being recorded in the format of logs. 

An event log is a set of traces, each representing the execution of a process (case instance). Each trace consists of a sequence of activities, each referring to the execution of an activity in a finite activity set $A$.
\begin{definition}[Trace, Event Log]
	A trace $\sigma=\left\langle a_1, a_2, ... a_n\right\rangle \in A^*$ over $A$ is a sequence of activities. An event log $L \in \cal{B}$(A) is a multi-set of traces over the activity set $A$.
\end{definition}

A \emph{prefix of length} $k$ of a trace $\sigma=\left\langle a_1, a_2, ... a_n\right\rangle \in A^*$, is a trace $p_k(\sigma)=\left\langle a_1, a_2, ... a_k\right\rangle \in A^*$ where $k \leq n$; the \emph{suffix of the prefix of length} $k$ is defined as the remaining part of $\sigma$, that is, $s_k(\sigma) = \left\langle a_k+1, a_k+2, ... a_n\right\rangle \in A^*$. For example, the prefix of length $3$ of $\npla{a,c,r,f,s,p}$ is $\npla{a,c,r}$, while the suffix of this prefix is $\npla{f,s,p}$.

A \emph{cycle} in a trace $\sigma \in A^*$ is a sequence of activities repeated at least twice in $\sigma$ (with adjacent repetitions). For example, trace $\npla{a,b,a,b,a,b,c,d,e,f,g,e,f,g,c,d}$ contains two cycles: $\npla{a,b}$ (3 repetitions) and $\npla{e,f,g}$ (2 repetitions).


\subsection{Process mining}

Process mining is a business process management technique that enables the analysis of processes based on the event logs. Main idea lies at improving business processes based on analysis of the logs as sequential, continuous processes. That what makes it different from the methods established in data mining and business intelligence which treat the data as set of distinct events to gather insights. 

As a rule the methods of process mining applied mostly to the processes that do not have formal description or have consistent deviations from the predicted in the system. 

The three main activities performed under the process mining field are:

\begin{itemize}
	\item \textbf{Process analysis (Discovery)} encompasses all methods to discover insights from the past process executions. 
	\item \textbf{Conformance analysis} is to take existent formalized model check it against real log
	\item \textbf{Process enhancement} helps to use log in order to enhance current executions of the processes
\end{itemize}
 


\subsection{Predictive monitoring}

Predictive monitoring is an advancing-sub field of Process Mining. As a a sub field of Process Mining, its main focus also lies in exploring and exploiting the process logs. More specifically, it deals with online predictions of process outcomes. Predicted outcomes of interest can be compliance of with a measure, achieving objective, or classifications. Classification is giving a label to the uncompleted trace.  
\par
Many algorithms are applied to mentioned problems, these include classic machine learning algorithms of classification and clustering. More complex concepts are also used, such as Markov models~\cite{Leontjeva2015}, time series, and deep learning\cite{niek96732,evermann,quteprints96732} approaches. 
\par
In order to successfully apply these algorithms for aforementioned problems, one usually needs to have solid understanding on the process log nature. The aspects that can come into play, are the data types, stationarity, or heteroscedasticity. More complications can be brought if log contains unstructured content such as textual comments.
\par
A log is a set of traces, each containing events with its even class, ID, name, and usually time stamp and n-dimensional payload vector. Events correspond to particular cases in the process. Early process mining approaches tend to ignore the data payload, taking into consideration only the control-flow (sequence of activities). Some of the approaches do the opposite, that is consider data, without control-flow~\cite{vanderAalst2010,DBLP:journals/is/AalstSS11,Schonenberg2008}  
\par
In that case one can say that the traces were considered to be a simple sequences. Moreover, methods introduced usually designed to work in offline fashion, exploring completed cases. But, to make more accurate, mature, online predictions one should abandon neither payload nor control-flow information. That's when the Complex Symbolic Sequences as methods of encoding of a log are introduced. 



\subsection{Complex Symbolic Sequences}

Every machine learning problem needs the data represented in a particular way, and algorithms adapted to the encoded data. For classical predictions of next steps in the sentence having bounded space of variants at each step the Simple Symbolic Sequence encoding is used. It is an encoding representing the sequence in a vector or a array of symbols.  

Simple Symbolic Sequences for log encoding can be used with a range of methods inherited from a massive number of works from machine learning and statistics such as concerning problems from health-informatics, or anomaly detection in Unix access systems. determining fraud users by its query log sequences.~\cite{Xing:2010:BSS:1882471.1882478} That is because the simple symbolic sequence encoding is simply an ordered list sampled out of some given alphabet. 

With the use of Simple Symbolic Sequences, it was relatively simple to apply methods, feature based classifications, sequence distance based, support vector machines, or model based classification~\cite{Xing:2010:BSS:1882471.1882478}.

Simple Symbolic Sequences have limitations that prevent getting best results of their use.
That is because simple symbolic sequences discard most of the rich information present in process logs, that is not limited to control-flow. An example would be the log in the big fast food chain, where the process about their employee's actions is daily recorded into the log. In Table 1, you can see that log of this nature contains static features, control-flow components and number of dynamic features. Static feature include ID of the employee, and his age. Dynamic features are the $t_i$ that is the time spent working, $rID_i$ is an ID of the restaurant that employee worked in on that day, $rev_i$ money he earned for the company. 

This log is rich with information additional information, and in order to work with it, one would want to capture as much information as possible.

On contrary to Simple Symbolic Sequences, the Complex Symbolic Sequences are introduced and we can capture much more information about log. Complex Symbolic Sequence is basically ordered list of vectors, where each vector is a subset of some alphabet inferred from the log.

\begin{table}[h]
	\centering
	\begin{tabular}{| l | l | l | l | l | l | l | l | l | l | l | l |}
		\hline
		& Age & ID & t\_1 & .. & t\_n & rID\_1 & .. & rID\_n & rev\_1 & .. & rev\_n \\	
		\hline
		tr1 & 20 & 00123 & 28800 & .. & 24840 & 11 & .. & 12 & 120\$ & .. & 112\$ \\
		tr2 & 24 & 00023 & 16890 & .. & 23460 & 12 & .. & 12 & 80\$ & .. & 111\$ \\
		tr3 & 18 & 01129 & 23804 & .. & 17897 & 11 & .. & 11 & 340\$ & .. & 23\$ \\
		tr4 & 21 & 00772 & 40034 & .. & 24564 & 12 & .. & 12 & 130\$ & .. & 2\$ \\
		\hline
		
	\end{tabular}
	\caption{Example of Complex Symbolic Sequence encoding}
	\label{tab:statements}
\end{table}



When Complex Symbolic Sequences are used instead of Simple Symbolic Sequences complexity space of the prediction problem grows substantially. So it suggests to use new and more complex approaches to work with them.

Some of the base lines of working with complex symbolic sequences use index-based encodings, and index latest payload encodings mentioned in following paper~\cite{Leontjeva2015}.   

\subsection{Structured and unstructured data}

There are few distinct types of data, that can be stored in the event. Simple, structured types include time-stamps, numbers, and nominal values. If the log contains only these types it is called a log with structured data. 

On the other hand, the logs can contain text information such as comments or emails, bringing up the complexity level of the problem. Text mining methods in conjunction with sequence classification should be used to bring down the problem~\cite{DBLP:conf/bpm/TeinemaaDMF16}.

As for example mentioned above, it might be a daily comment from manager of the employees in the fast-food chain. 

Also, as for example of the issue to resolution process, textual comments on performance of the employee are added to events. In the repair agency, the comments can be added with registering of the issue by customer (describing the initial assessment of the issue), and later in repair process by any involved party, that are repairmen, managers, or customers. Also comments can be some content of the emails what are associated with a business process in the company. 
\\

With this knowledge about types of business processes it is easier to understand the limitations of the current state of the art methods to be described in the related work section. 


\subsection{Hidden Markov Models}
Hidden Markov Models (HMMs) are one of the most used probabilistic models. These models belong to the family of directed probabilistic graphical models. 

As for the model, random variables are divided into two sets, namely a set seen outside, an \textit{observations} set $X=X_1,...,X_i,...,X_n$, and a hidden sequence (that is considered to be a first order Markov chain) of \textit{states} $Y=Y_1,...,Y_i,...,Y_n$.

The model is built by four sets of parameters. 

\begin{center}
	\centering
	\begin{tabular}{| l | l  |}	
		\hline
		$P(Y_1 = c_k | Y_0 = start) $, & initial probability \\
		\hline
		$P(Y_i = c_k | Y_{i-1} = c_l) $ & transition probability \\
		\hline
		$P(Y_i = stop | Y_{i-1} = c_k) $ & final probability \\
		\hline
		$P(X_i = w_j | Y_{i} = c_k) $ & emission probability \\
		\hline
	\end{tabular}
\end{center}

The observation set is the one that is seen on the outside. The usual usage of HMMs is having an HMM defined by parameters above to predict the sequence of states.

In order to predict the sequence the Viterbi algorithm is used. It is a perfect example of a dynamic programming algorithm to find \textit{max-product} of probabilities over all pace of sequences.

\subsection{Conditional Random Field}

The HMM model described above is a generative model, that is used to model the joint distribution of the outcome and the features $P(Y,X)$. There is another version of sequential models that is discriminative. It is called Conditional Random Field. It is used to build a conditional probability $P(Y|X)$ of the outcome.

CRF is a random field, in which the output $y$ is conditioned on input $x$. The conditional probability takes form

\[P(Y|X)=1/Z(X) * \exp(\sum_{m=1}^{M}{(\lambda_m * f_m (y_n, y_{n-1}, x_n))})\]

where the $Z(X)$ normalization function of the form 

\[Z(X)=\sum_{y}{\exp(\sum_{m=1}^{M}{(\lambda_m * f_m (y_n, y_{n-1}, x_n))})}\]

The main advantage of this model is relaxation of condition of observed data with labels. 




\subsection{RNNs and LSTM}
Artificial Neural Networks (or just Neural Networks, NNs) are a well
known class of discriminative models. In classification tasks, they are
used to model the probability of a given input to belong to a certain
class, given some features of the input. We can describe them in
mathematical terms as follows:
%
\begin{equation}
  \label{eq:condprob}
  p(\mathbf{y}|\mathbf{x}) = f_{\mathit{NN}}(\mathbf{x}; \theta).
\end{equation}
%
In \eqref{eq:condprob}, $\mathbf{x}$ is the feature vector that represents the input,
$\mathbf{y}$ is a random variable representing the output class
labels, $f_{\mathit{NN}}$ is the function modeled by the neural network, and
$\theta$ is the set of parameters of such a function to be learnt during
the training phase.


\begin{figure}[t]
  \centering
  \resizebox{.56\textwidth}{!}{
    \begin{tikzpicture}[node distance=5em]
	% \matrix [column sep={6em,between origins},row sep=4em]
	% {
      \node [name=x_1] {$\mathbf{x}^{\langle 1 \rangle}$};
      \node [name=x_2, right of=x_1] {$\mathbf{x}^{\langle 2 \rangle}$};
      \node [name=x_3, right of=x_2] {$\mathbf{x}^{\langle 3 \rangle}$};
      \node [name=x_dots, right of=x_3, xshift=-1em] {\ldots};
      \node [name=x_K, right of=x_dots, xshift=-1em] {$\mathbf{x}^{\langle K \rangle}$};

      \node [draw, shape=circle, name=h_1, above of=x_1, yshift=-1.5em, label=above right:{$\mathbf{h}^{\langle 1 \rangle}$}] {};
      \node [draw, shape=circle, name=h_2, above of=x_2, yshift=-1.5em,label=above right:{$\mathbf{h}^{\langle 2 \rangle}$}] {};
      \node [draw, shape=circle, name=h_3, above of=x_3, yshift=-1.5em,label=above right:{$\mathbf{h}^{\langle 3 \rangle}$}] {};
      \node [name=h_dots, above of=x_dots, yshift=-1.5em] {\ldots};
      \node [draw, shape=circle, name=h_K, above of=x_K, yshift=-1.5em, label=above right:{$\mathbf{h}^{\langle K \rangle}$}] {};

      \path [draw, -latex'] (x_1.north) -- (h_1.south);
      \path [draw, -latex'] (x_2.north) -- (h_2.south);
      \path [draw, -latex'] (x_3.north) -- (h_3.south);
      \path [draw, -latex'] (x_K.north) -- (h_K.south);

      \node [name=y_1, above of=h_1, yshift=-1.5em] {$\mathbf{y}^{\langle 1 \rangle}$};
      \node [name=y_2, above of=h_2, yshift=-1.5em] {$\mathbf{y}^{\langle 2 \rangle}$};
      \node [name=y_3, above of=h_3, yshift=-1.5em] {$\mathbf{y}^{\langle 3 \rangle}$};
      \node [name=y_dots, above of=h_dots, yshift=-1.5em] {\ldots};
      \node [name=y_K, above of=h_K, yshift=-1.5em] {$\mathbf{y}^{\langle K \rangle}$};

      \path [draw, -latex'] (h_1.north) -- (y_1.south);
      \path [draw, -latex'] (h_2.north) -- (y_2.south);
      \path [draw, -latex'] (h_3.north) -- (y_3.south);
      \path [draw, -latex'] (h_K.north) -- (y_K.south);

      \path [draw, -latex'] (h_1.east) -- (h_2.west);
      \path [draw, -latex'] (h_2.east) -- (h_3.west);
      \path [draw, -latex'] (h_3.east) -- (h_dots.west);
      \path [draw, -latex'] (h_dots.east) -- (h_K.west);
      % }
    \end{tikzpicture}
  }
  \caption{Recurrent Neural Network}
  \label{fig:rnn}
\end{figure}

Recurrent Neural Networks (RNNs, see Fig.~\ref{fig:rnn}) are a subclass of Neural Networks. We
illustrate them with the help of an example in which the
classification task concerns in assigning the correct part of speech -- noun, verb, adjective, etc. --
to words.  If we take the word \emph{``file''} in isolation,
it can be both a noun and a verb. Nonetheless, this ambiguity
disappears when we consider it in an actual sentence. Therefore, in the
sentence \emph{``I have to file a complain''} it acts as a verb, while
in the sentence \emph{``I need you to share that file with me''} it
acts as a noun.

This simple example shows that for some tasks the classification at a
certain time-step $t$ depends not only on the current input (i.e.,
\emph{``file''}) but also on the input (i.e., the part of the
sequence) seen so far. The tasks that share this characteristic are
said to be \emph{recurrent}. Natural Language tasks are a typical
example of recurrent phenomena.

In mathematical terms, let us write
$\mathbf{x}^{\langle 1 \rangle}, ..., \mathbf{x}^{\langle K \rangle}$
to indicate an input sequence of $K$ time-steps, represented by the
superscript between angle brackets. In this way, at each time-step $t$,
the conditional probability of a given input to belong to a certain class is described by
%
\begin{equation}
\label{eq:condprobrec}
  p(\mathbf{y}^{\langle y \rangle}|\mathbf{x}^{\langle t \rangle}, ..., \mathbf{x}^{\langle 1 \rangle}) = f_{\mathit{RNN}}(\mathbf{x}^{\langle 1 \rangle}, ..., \mathbf{x}^{\langle t \rangle}; \theta).
\end{equation}
%
RNNs have been proven to be extremely appropriate for modeling sequential
data (see~\cite{goodfellow2016dlbook}). As shown in Fig.~\ref{fig:rnn}, they typically leverage
recurrent functions in their hidden layers, which are, in turn, composed of hidden states. Let
$\mathbf{h}^{\langle t \rangle}$, with
\begin{equation}
  \label{eq:hidden}
  \mathbf{h}^{\langle t \rangle} = h(\mathbf{x}^{\langle t \rangle}, \mathbf{h}^{\langle t-1 \rangle}; \theta_{h});
\end{equation}
be the activation of the hidden state at the $t$-th time-step. $h$ is a so-called \emph{cell function}, parameterized over a set
of parameters $\theta_{h}$ to be learnt during the training, and accepting as inputs the current input $\mathbf{x}^{\langle t \rangle}$ and its value at the
previous time-step $\mathbf{h}^{\langle t-1 \rangle}$.
The activation of the hidden state is then mapped (using a linear map) into a continuous vector of the
same size as the number of output classes. All the elements in such a vector are greater than zero and their sum is equal to one. Therefore, this vector can be seen as a probability distribution over the output space.
All these constraints can be easily achieved specifying the generic equation \eqref{eq:condprobrec} by means of a softmax function:
%
\begin{equation}
  \label{eq:softmax}
  p(\mathbf{y}^{\langle y \rangle}|\mathbf{x}^{\langle t \rangle}, ..., \mathbf{x}^{\langle 1 \rangle}) =
  softmax(\mathbf{W}\mathbf{h}^{\langle t \rangle} + \mathbf{b});
\end{equation}
%
where the weight matrix $\mathbf{W}$ and the bias vector $\mathbf{b}$
are parameters to be learnt during the training phase.

% The presentation of RNNs ends with a note about the representation of the input: at each time-step, the
% input should be represented by means of a vector. When our input space is
% made of a finite set of symbols (e.g. a vocabulary of words) the most
% straightforward way to represent it is to use a one-hot encoding:
% the $i$-th symbol will be encoded into a vectors of dimension equal to
% the dimension of the input space with all the elements set to 0 and
% just the $i$-th to one. Such representations are not very practical:
% they grow linearly with the dimension of the input space; their sum produce a vector which is not one-hot anymore; multiplying them
% element-wise produce an all-zero vector; and so on. To overcome such limitations, the so called \emph{low
%   dimensional embedding} technique is used: the $i$-th symbol is
% represented with the $i$-th row of an \emph{embedding matrix} learnt
% at training time. The number of the column of the matrix, and
% consequently the dimension of the input vectors, is an hyper-parameter
% of the model to be set at design time.

Among the different cell functions $h$ (see equation~\eqref{eq:hidden}) explored in literature, Long Short-Term
Memory (LSTM)~\cite{hochreiter1997long} shows a significant ability to maintain the memory of its input across long time spans. This property makes them extremely suitable to be used in RNNs that have to deal with input sequences with complex long-term dependencies such as the ones we consider in this paper.

% The cell made of four \emph{gates}, namely the input gate $\mathbf{i}$, the forget gate $\mathbf{f}$, the output gate $\mathbf{o}$ and the cell update gate $\mathbf{g}$.
% At each time step, we have that:
% \begin{equation}
%   \label{eq:lstmgates}
%   \begin{aligned}
%     \mathbf{i}^{\langle t \rangle} &= \sigma(\mathbf{W}_i\mathbf{x}^{\langle t \rangle} + \mathbf{U}_i\mathbf{h}^{\langle t-1 \rangle} + \mathbf{b}_i);\\
%     \mathbf{f}^{\langle t \rangle} &= \sigma(\mathbf{W}_f\mathbf{x}^{\langle t \rangle} + \mathbf{U}_f\mathbf{h}^{\langle t-1 \rangle} + \mathbf{b}_f);\\
%     \mathbf{o}^{\langle t \rangle} &= \sigma(\mathbf{W}_o\mathbf{x}^{\langle t \rangle} + \mathbf{U}_o\mathbf{h}^{\langle t-1 \rangle} + \mathbf{b}_o);\\
%     \mathbf{g}^{\langle t \rangle} &= f(\mathbf{W}_g\mathbf{x}^{\langle t \rangle} + \mathbf{U}_g\mathbf{h}^{\langle t-1 \rangle} + \mathbf{b}_g);\\
%   \end{aligned}
% \end{equation}
% where
% $\theta_{LSTM} = [\mathbf{W}_i, \mathbf{U}_i, \mathbf{b}_i,
% \mathbf{W}_f, \mathbf{U}_f, \mathbf{b}_f, \mathbf{W}_o, \mathbf{U}_o,
% \mathbf{b}_o, \mathbf{W}_g, \mathbf{U}_g, \mathbf{b}_g]$ is the set of
% parameters to be learned during the training phase.  The cell holds an
% internal state $\mathbf{c}^{\langle t \rangle}$ that is updated at
% each timespep according to the following:
% \begin{equation}
%   \label{eq:lstmstate}
%   \mathbf{c}^{\langle t \rangle} = \mathbf{f}^{\langle t \rangle} \odot \mathbf{c}^{\langle t-1 \rangle} + \mathbf{i}^{\langle t \rangle} \odot \mathbf{g}^{\langle t-1 \rangle};
% \end{equation}
% being $\odot$ the element-wise product between two vectors,
% $\sigma(\cdot)$ the sigmoid function and $f(\cdot)$ a non-linear
% function, typically the hyperbolic tangent $\tanh(\cdot)$. Intuitively
% we can see how the forget gate $\mathbf{f}$ controls how much of the
% previous cell state flows into the current state, while the input gate
% controls how much of the cell update signal gets into it. Finally, the
% cell activation is given by the:
% \begin{equation}
%   \label{eq:lstmactivation}
%   \mathbf{h}^{\langle t \rangle} = \mathbf{o}^{\langle t \rangle} \odot f(\mathbf{c}^{\langle t \rangle});
% \end{equation}
% where the output gate controls the fraction of output signal coming from the cell state.

\subsection{RNNs with LSTM for Predictive Process Monitoring}
\label{subsec:RNNforpredictive}
In order to provide predictions on the suffix of a given prefix (of a running case), state-of-the-art approaches for predictive process monitoring use RNNs with LSTM cells.  The most recent and performing approach in this field \cite{niek96732} relies on an encoding of activity sequences that combine features related to the activities in the sequence (the so called \textit{one-hot encoding}) and features related to the time characterizing these activities. Given the set $A = \{a_{1_A}, \ldots a_{m_A}\}$ of all possible activities, an ordering function $idx:A \rightarrow \{1, \ldots, \left|A\right| \} \subseteq \mathbb{N}$ is defined on it, such that $a_{i_A}<>a_{j_A}$ if and only if $i_A<>j_A$, i.e., two activities have the same A-index if and only if they are the same activity. For instance, if $A=\{a,b,c\}$, we have $idx: A \rightarrow \{1,2,3\}$ and $idx(a)=1$, $idx(b)=2$ and $idx(c)=3$. Each activity $a_i \in \sigma$ is encoded as a vector $\mathbb(A_i)$ of length $|A|+3$ such that the first $|A|$ features are all set to $0$, except the one occurring at the index of the current activity $idx(a_i)$, which is set to $1$. The last three features of the vector pertain to time: the first one relates to the time increase with respect to the previous activity, the second reports the time since midnight (to distinguish between working and night time), and the last one refers to the time since the beginning of the week.
%For example, given the alphabet $A$ above, the activity $b$ in a trace $\sigma =  \left\langle bcc \right\rangle$ will be encoded as the vector $(0,1,0)$.

A trace is encoded by composing the vectors obtained from all activities in the trace into a matrix. During the training phase, the encoded traces are used for building the LSTM model. During the testing phase, a (one-hot encoded) prefix of a running case is used to query the learned model, which returns the predicted suffix by running an inference algorithm.
%\todocdf{Do we need other details about the architecture?}
Algorithm~\ref{alg:baseline} reports the inference algorithm introduced in \cite{niek96732} and based on RNN with LSTM cells for predicting the suffix of a given prefix $p_k(\sigma)$ of length $k$. The algorithm takes as input the prefix $p_k(\sigma)$, the LSTM model $lstm$ and a maximum number of iterations $max$ and returns as output the complete trace (the prefix and the predicted suffix).
First, the prefix $p_k(\sigma)$ is encoded by using the one-hot encoding (line~\ref{lst1:encoding}). The resulting matrix is then used for feeding the LSTM model and getting the probability distribution over different possible symbols that can occur in the next position of the trace (line~\ref{lst1:prob}). The symbol with the highest probability is hence selected from the ranked probabilities (line~\ref{lst1:next}). Then, a new trace is obtained by concatenating the current prefix with the new predicted symbol (line~\ref{lst1:trace}). In order to predict the second activity, the one-hot encoding of the new prefix is computed and used to recursively feed the network. The procedure is iterated until the predicted symbol is the end symbol or a maximum number of iterations $max$ is reached (line~\ref{lst1:while}).

\begin{algorithm}
	\caption{Inference algorithm for predicting the suffix of $p_k(\sigma)$}
	\label{alg:baseline}
	\begin{algorithmic}[1]
		\Function{PredictSuffix}{$p_k(\sigma)$, $lstm$, $max$}
		\State $k$ = 0
		\State $trace$ = $p_k(\sigma)$
		\Do
		\State $trace_{encoded}$ = \textsc{encode}($trace$) \label{lst1:encoding}
		\State $next\_symbol\_probs$ = \textsc{predictNextSymbols}($lstm$, $trace_{encoded}$) \label{lst1:prob}
		\State $next\_symbol$ = \textsc{getSymbol}($next\_symbol\_prob$, $trace_{encoded}$) \label{lst1:next}
		\State $trace$ = $trace \cdot next\_symbol$\label{lst1:trace}
		\State $k = k + 1$
		\doWhile{$($not $next\_symbol$ == $end\_symbol)$ and $(k$ \textless $max)$}\label{lst1:while}
		\State \textbf{return} $trace$
		\EndFunction
	\end{algorithmic}
\end{algorithm}

%\begin{algorithm}[H]
%	\KwData{$p_k(\sigma)$}
%	\KwResult{$\sigma$}
%	initialization\;
%	\While{end symbol not predicted}{
%		enc = encode-one-hot ($p_k$)\;
%		prob-distr = model.predict(enc)\;
%		predicted = choose most probable(prob-distr)\;
%		$p^{k+1}$ = concatenate $p_k(\sigma)$ + predicted\;
%		k = k + 1\;
%	}
%	return $p_k$
%	\caption{Inference in RNN for predicting the suffix of a prefix of $k$ symbols}
%	\label{alg:baseline}
%\end{algorithm}

%\todocdf{Which one shall we keep?}

%\begin{algorithm}
%	\caption{Baseline inference algorithm for predicting suffix of activities}
%	\label{alg:baseline}
%	\begin{algorithmic}[1]
%		\Function{EvaluateSuffix}{$prefix$, $model$, $limit$}
%		\State $k$ = 0
%		\State $traceResulting$ = $prefix$
%		\Do
%		\State $encoded$ = encode($traceResulting$)
%		\State $nextSymbolProb$ = predictNext($model$,$encoded$)
%		\State $nextSymbol$ = getSymbol($nextSymbolProb$)
%		\State $traceResulting$ = concatenate($traceResulting$, $nextSymbol$)
%		\State k = k + 1
%		\doWhile{$($not $nextSymbol$ equal to $endSymbol)$ and $(k$ \textless $limit)$}
%		\State return traceResulting
%		\EndFunction
		
%		\Function{getSymbol}{$nextSymbolProb$}
%		\State top = chooseMaxProb($nextSymbolProb$)
%		\State return probabilityToSymbol(top)
%		\EndFunction
		
%	\end{algorithmic}
%\end{algorithm}




\subsection{Linear Temporal Logic}
In our approach, the a-priori knowledge that describes how a running case will develop in the future is formulated in terms of Linear Temporal Logic (LTL) rules~\cite{Pnueli77}. LTL is a modal logic with modalities devoted to describe time aspects. Classically, LTL is defined for infinite traces. However, to describe the characteristics of a business process, we use a variant of LTL defined for finite traces (since business processes are supposed to complete eventually).
%
We assume that activities occurring during the process execution fall into the set of atomic propositions. LTL rules are constructed from these atoms by applying the temporal operators $\NEXT$ (next), $\sometime$ (future), $\always$ (globally), and $\until$ (until) in addition to the usual boolean connectives. Given a formula $\varphi$, $\NEXT \varphi$ means that the next time instant exists and $\varphi$ is true in the next time instant (strong next). $\sometime \varphi$ indicates that $\varphi$ is true sometimes in the future. $\always \varphi$ means that $\varphi$ is true always in the future. $\varphi \until \psi$ indicates that $\varphi$ has to hold at least until $\psi$  holds and $\psi$ must hold in the current or in a future time instant.

\subsection{Beam search}

Beam search is a heuristic search algorithm, based on the principle of exploring the graph of possible solutions by the most promising node. It is built on the assumption of huge search space. It orders all partial solutions with some heuristic measure keeping only small portion of the most promising ones, expanding them to get the complete solution.

There are few variations of this algorithm, from which two will be thoroughly explained here.

\subsubsection{Best-first variation of Beam search}

Best-first search explores the solution space following the most probable solution first. The algorithm exploits a tree structure, in which each node maintains its state as open and closed. The algorithm assumes that all nodes are open at the initialization, and after visiting the nodes marks them as closed. 

The pseudo code for the algorithm is following:


\begin{algorithm}
	\caption{Best first search}
	\label{alg:protrack}
	\begin{algorithmic}[1]
		\State define node of the graph, OPEN, to be starting node $s$
		\If {\textsc{list} is empty} 
		\State	failure
		\EndIf
		\State Take the best node from the list (call it $n$), and make it CLOSED
		\State Expand node $n$
		\If {any of the successor nodes are the solution} 
			\State	success, return the solution traced back
		\EndIf
		\ForAll {node of the successors} 
		\State	apply evaluation function $f$ to the $node$\;
		\State	if $node$ is new (not CLOSED) mark as OPEN\;
		\EndFor
	\end{algorithmic}
\end{algorithm}



\vspace{0.2 cm}

Different evaluation function can be used for the task. The decision is made upon the specifics of the problem, and it directly influences the effectiveness of the algorithm. 

\subsubsection{Breath-first variation of Beam search}

Breath-first search begins with an empty node and explores the search space in the breath-wise order. 

This type of beam search operates using few priority queues instead of tree structures as in previous approach. 

Here is the pseudo code for the algorithm (all elements in priority queues are stores with priority based on some evaluation function):

\begin{algorithm}
 	\caption{Breath-first Beam search}
	\label{alg:protrack}
	\begin{algorithmic}[1]
		\State define two empty priority queues, $s1$ and $s2$
		\State add the first element to $s1$
		\While {solution is not found or the search space is not empty} 
			\While {s1 is not empty}
				\State get element $elem$ from s1
				\If{ solution found } 
					\State	success
					\EndIf
					\State expands $elem$ and write $beamsize$-elements to s2
					\EndWhile
					\State $s2$ equals first part of $s1$ with the size of $beamsize$
					\EndWhile
	\end{algorithmic}
\end{algorithm}




\vspace{0.2 cm}
Both of these algorithms use the same tactic by cutting the space to the most probable and better performing prospective solutions. These algorithms are different, and depending on the nature of the problem given can have very different performance.

Beam search is used in many areas such as machine translation\cite{Koehn2007MOSEStranslationSystem}.


% section background (end)

%%% Local Variables:
%%% mode: latex
%%% TeX-master: "BPM17"
%%% End: 



