%  tlsymbols.tex
%
        %%%%%%%%%%%%%%%%%%%%
        %     WARNING      %
        %%%%%%%%%%%%%%%%%%%%
% The master copy of this file is the one that resides in
% /usr/sun3/common/lib/tex/macros
% If you modify any other copy of the file it will be overwritten by
% the nightly rdist ditribution
%

% First input tlformat.tex:

% The mdf sponsored, miw designed macro to put text back in
% roman in while in math mode:
%
%Here are some quotes:" % prints quotes
%
%And in math mode, $x="some roman text"$.
{\catcode`\"=\active
\gdef\turnonquotes{\catcode`\"=\active \let"=\dobox}
\gdef\dobox#1"{\hbox{\rm #1}}}

\everymath={\turnonquotes}
\everydisplay={\turnonquotes}


% The signature of functions can be done using the following 2 macros
%
\def\funcsig#1#2{#1\rightarrow{#2}}
\def\Signature#1#2{#1\colon\, #2}
\let\Sig=\Signature

% Put a box around some text:
\def\parlinebox#1{\begin{center}
                  \fbox{\parbox{4.5in}{#1}}
                  \end{center}}

% To put a silly quote at the beginning of your text, do:
% \sillyquote{Some text}{By whom} right at the beginning of your
% chapter.
\def\sillyquote#1#2{%
        \begin{flushright}
        \parbox[t]{3in}{{\noindent #1}
                
        \def\baselinestretch{1}
        {\it #2}}
        \end{flushright}
        \vspace{1cm}}

\def\vsillyquote#1#2{%
        \begin{flushright}
        \parbox[t]{3in}{{\noindent #1}
                
        \def\baselinestretch{1}
        \begin{flushright}{\it #2}\end{flushright}}
        \end{flushright}
        \vspace{1cm}}

%To put an old-fashioned centerline across the page:
\def\mycentreline{\begin{center} \rule{3.8in}{0.005in}\end{center}}

% Now on to `proper' semantics!
% We've found that sometimes (e.g., using crappy printers that blur our 
% symbols) the obvious use of \! in [\! [ etc moves the 
% brackets a little too close together, so we have:

\def\squash{\mskip-2.5mu}
% and we can replace all obvious occurences of \! by \squash ...
% That's if you like it that way.
% (obviously this can be changed easily back to 3 or 2.5 etc.,..

\def\leftsem{\lbrack\!\lbrack}
\def\rightsem{\rbrack\!\rbrack}
% these two definitions are used consistently, so if a better way 
% is found of doing these brackets (e.g., pictures) just substitute
% in these two and the rest will follow

\def\SEMt#1{\hbox{$\leftsem #1\rightsem $}}
\def\SEM#1{\leftsem #1\rightsem}
\def\DENSEM#1{{\cal D}\leftsem #1\rightsem}
\def\OPSEM#1{{\cal O}\leftsem #1\rightsem}
\def\SSSEM#1{{\cal S}\leftsem #1\rightsem}
\def\GENSEM#1#2{{\cal #1}\leftsem #2\rightsem}
\def\TSEM#1{{\cal T}\leftsem #1\rightsem}

% Now for some macros useful for talks on temporal semantics!
% First, one of Grahams
\def\Vlongline{\hbox to 140pt{\hrulefill}}
\newcommand{\infrule}[2]{{\renewcommand{\arraystretch}{0.6}\begin{array}{c} #1 \\       \Vlongline \\#2 \\ \end{array}}}
% and now for michael's hack of grahams beautiful makerow ...
\newcommand{\superinfrule}[3]{{\renewcommand{\arraystretch}{0.6}\begin{array}{#1} #2 \\ \Vlongline \\#3 \\ \end{array}}}

% and for a more general version of both these:
% The following macro takes 4 arguments. These are interpreted as follows:
% #1 --- defines the positioning of the array. (N.B., if other than l,
%        c, or r are used, `&'s must be in the arguments somewhere
% #2 --- argument for the top element of the inference rule
% #3 ---   "       "   "  bottom element.
% #4 --- optional arraystretch parameter
%
\newcommand{\inferencerule}[4]{\renewcommand{\arraystretch}{#4}\begin{array}{#1}{}#2\\\hline{}#3\end{array}}

% Now one of Graham's environments to make comments stand out:
%
% \newenvironment{comment}%
%         {\begin{center}\begin{minipage}%
%                         {.9\textwidth}\small\tt}%
%                       {\end{minipage}%
%          \end{center}}

% Now a couple of environments for doing arrays in square brackets:
%
\newenvironment{sqarray}[2]{\left[ \begin{array}[#1]{#2}}{\end{array} \right]}
                        


\makeatletter
\def\@psfmtname{pslplain}
\ifx\fmtname\@psfmtname \else \def\cmsy@{2}\fi % make sure we can always get cmsy

% This file contains a few LaTeX definitions for temporal symbols.
% Most are standard, the first few being generated using LaTeX picure
% mode.
% Section 1 contains mainly common temporal symbols, in particular:
%       \always                 --- the standard `box' symbol
%       \sometime               --- the standard `diamond' symbol
%       \next                   --- the open circle (weak or standard next)
%       \snext                  --- open circle with a dot in the
%                                   middle (strong next operator)
%       \last                   --- filled in circle ( standard last)
%       \slast                  --- as \last, but with small white
%                                   circle in centre (strong last)
%       \until                  --- standard binary (strong) until operator
%                                   ( calligraphic `U')
%       \since                  --- standard binary (strong) since operator
%                                   ( calligraphic `S')
%       \chop                   --- standard binary chop operator
%                                   (calligraphic `C')
%       \ctlFinf                --- CTL* `infinitely often operator'
%                                   (F with an infinity on top)
%       \ctlGalm                --- CTL* `almost always operator'
%                                   (G with an infinity on top)
%       \infoften               --- (\sometime with an infinity on top)
%       \almalways              --- (\always with an infinity on top)
%
%
% Section 2.    And now for some non-temporal but very standard operators:
%               ----------------------------------------------------------
%       \Nat                    --- standard natural numbers symbol
%       \Int                    --- standard integers symbol
%       \Real                   --- standard real numbers symbol
%       \Ratl                   --- standard rational numbers
%       \Bool                   --- you guessed it!
%
% Section 3 contains more operators for different systems, e.g.,
% CSP, CCS, Dynamic Logic, Process Logic, etc.,
%
% So, here we go:
%
% Section 1: Temporal Symbols
% ===========================
%
%
% 1a: Yer basic temporal operators:
%     ----------------------------
%
%\newcommand{\always}{\raisebox{-.2ex}{
 %                          \mbox{\unitlength=0.9ex
  %                         \begin{picture}(2,2)
   %                        \linethickness{0.06ex}
    %                       \put(0,0){\line(1,0){2}}
     %                      \put(0,2){\line(1,0){2}}
      %                     \put(0,0){\line(0,1){2}}
       %                    \put(2,0){\line(0,1){2}}
        %                   \end{picture}}}
         %             \,}

\newcommand{\always}{\square}

\newcommand{\alwaysP}{\rule[-0.2ex]{1.8ex}{1.8ex}\,}
%\def\sometime{\mathord{\hbox{\large$\mathchar"0\cmsy@7D$}}} % PS \diamondsuit
                                                            % is too small
\def\sometime{\lozenge}

\ifx\fulldiamondsuit\undefined %
\def\sometimeP{\!\,\makebox[1em]{$\sometime$}\hspace{-1em}
    \raisebox{.25ex}{\makebox[1em]{$\bullet$}}\,} %
\else %
\def\sometimeP{\!\,\raisebox{-0.2ex}{\makebox[1em]{\LARGE$\fulldiamondsuit$}}}%
\fi
% make sure we can always get \sometimeP!!
\let\was=\sometimeP
%
%\newcommand{\sometime}{\,\raisebox{-.2ex}{
%                       \mbox{\unitlength=0.9ex
%                       \begin{picture}(2,2)
%                       \linethickness{0.06ex}
%                       \put(0.75,0){\line(3,4){0.75}}
%                       \put(0.75,0){\line(-3,4){0.75}}
%                       \put(0.75,2){\line(-3,-4){0.75}}
%                       \put(0.75,2){\line(3,-4){0.75}}
%                       \end{picture}}}
%                       \,}
%
%\newcommand{\sometimet}{\,\raisebox{-.2ex}{
%                       \mbox{\unitlength=0.9ex
%                       \begin{picture}(2,2)
%                       \linethickness{0.06ex}
%                       \put(0.64,0){\line(2,3){0.64}}
%                       \put(0.64,0){\line(-2,3){0.64}}
%                       \put(0.64,2){\line(-2,-3){0.64}}
%                       \put(0.64,2){\line(2,-3){0.64}}
%                       \end{picture}}}
%                       \,}
%
% But until the above work, we'll use the old favourite:
\def\mysometime{\hspace{-1.2ex}\hbox{\large$\mathchar"0\cmsy@7D$}}
\def\mysometimesmall{\hbox{$\mathchar"0\cmsy@7D$}}

%\newcommand{\NEXT}{\!\raisebox{-.2ex}{ %possibly add a little space before
 %                       \mbox{\unitlength=0.9ex
  %                      \begin{picture}(2,2)
   %                     \linethickness{0.06ex}
    %                    \put(1,1){\circle{2}} % Draws circle with
     %                   \end{picture}}}       % diameter 2 at centre 1,1
      %                  \,}

\newcommand{\NEXT}{\bigcirc}

\newcommand{\snext}{\raisebox{-.2ex}{ %possibly add a little space before
                        \mbox{\unitlength=0.9ex
                        \begin{picture}(2,2)
                        \linethickness{0.06ex}
                        \put(1,1){\circle{2}} % Draws circle with
                        % diameter 2 at centre 1,1, and puts dot in middle
                        \put(1,1){\circle*{0.4}}
                        \end{picture}}}
                        \,}

\newcommand{\last}{\!\raisebox{-.2ex}{
                        \mbox{\unitlength=0.9ex
                        \begin{picture}(2,2)
                        \linethickness{0.06ex}
                        \put(1,1){\circle*{1.92}} % Draws a filled in circle
                        \end{picture}}}       % with diameter 2 at centre 1,1
                        \,}
%OLD ONE:
%\newcommand{\slast}{\,\raisebox{-.2ex}{
%                       \mbox{\unitlength=0.9ex
%                       \begin{picture}(2,2)
%                       \linethickness{0.9ex}   % which doesn't matter
                                                % in picture mode when
                                                % drawing circles!
%                       \put(1,1){\circle{0.6}} % Draws a circle with%
                                                % a very thick line
%                       \put(1,1){\circle{0.8}}
%                       \put(1,1){\circle{1.0}}
%                       \put(1,1){\circle{1.2}}
%                       \put(1,1){\circle{1.4}}
%                       \put(1,1){\circle{1.6}}
%                       \put(1,1){\circle{1.8}}
%                       \end{picture}}}
%                       \,}
\newcommand{\slast}{\raisebox{-.2ex}{
                        \mbox{\unitlength=0.9ex
                        \begin{picture}(2,2)
                        \linethickness{0.9ex}   % which doesn't matter
                                                % in picture mode when
                                                % drawing circles!
%                       \put(1,1){\circle{0.6}} % Draws a circle with%
                                                % a very thick line
                        \put(1,1){\circle{0.9}}
                        \put(1,1){\circle{1.0}}
                        \put(1,1){\circle{1.2}}
                        \put(1,1){\circle{1.4}}
                        \put(1,1){\circle{1.6}}
                        \put(1,1){\circle{1.8}}
                        \put(1,1){\circle{1.86}}
                        \put(1,1){\circle{1.92}}
                        \end{picture}}}
                        \,}
%
% Now general macros that can be changed...
\def\stronglast{\!\!\slast}
\def\heretofore{\,\alwaysP}
\let\weaklast=\last
% The next two are like this becuae we assume infinite futures:
\let\strongnext=\next
\let\weaknext=\next
%
% Also we have Until and Chop symbols. Due to the confusion over what
% an Unless symbols should look like, we're going to leave that up to
% you. Some people use an italic `U', while others use a calligraphic `W':
% \def\unless{\hbox{$\, U\,$}}
% \def\unless{\hbox{$\,{\cal W}\,$}}
%
%\def\until{\hbox{$\,\mathchar"2255\,$}}
%\def\since{\hbox{$\,\mathchar"2253\,$}}
%\def\chop{\hbox{$\,\mathchar"2243\,$}}
\def\until{\sqcup}
%\def\unless{\hbox{$\, U\,$}}
% Now we've "plumped" for having unless as W!
\def\unless{\hbox{$\,\cal W \,$}}
\def\since{\hbox{$\,\cal S \,$}}
\def\zince{\hbox{$\,\cal Z \,$}}
\def\chop{\hbox{$\,\cal C \,$}}
\def\itchop{\hbox{$\, C\, $}}
\def\ituntil{\hbox{$\, U\, $}}

\newcommand{\alwaysf}{\raisebox{-.2ex}{
                           \mbox{\unitlength=0.9ex
                           \begin{picture}(2,2)
                           \linethickness{0.06ex}
                           \put(0,0){\line(1,0){2}}
                           \put(0,2){\line(1,0){2}}
                           \put(0,0){\line(0,1){2}}
                           \put(2,0){\line(0,1){2}}
                           \put(1,0.6){\line(0,1){0.8}}
                           \put(0.6,1){\line(1,0){0.8}}
                           \end{picture}}}
                      \,}

\newcommand{\alwaysb}{\raisebox{-.2ex}{
                           \mbox{\unitlength=0.9ex
                           \begin{picture}(2,2)
                           \linethickness{0.06ex}
                           \put(0,0){\line(1,0){2}}
                           \put(0,2){\line(1,0){2}}
                           \put(0,0){\line(0,1){2}}
                           \put(2,0){\line(0,1){2}}
                           \put(0.6,1){\line(1,0){0.8}}
                           \end{picture}}}
                      \,}

\newcommand{\alwaysfb}{\raisebox{-.2ex}{
                           \mbox{\unitlength=0.9ex
                           \begin{picture}(2,2)
                           \linethickness{0.06ex}
                           \put(0,0){\line(1,0){2}}
                           \put(0,2){\line(1,0){2}}
                           \put(0,0){\line(0,1){2}}
                           \put(2,0){\line(0,1){2}}
                           \put(1,0.9){\line(0,1){0.8}}
                           \put(0.6,1.3){\line(1,0){0.8}}
                           \put(0.6,0.7){\line(1,0){0.8}}
                           \end{picture}}}
                      \,}

%
% BEWARE: The following macros for sometime require pspicture option
%
\newcommand{\sometimef}{\,\raisebox{-.2ex}{
                        \mbox{\unitlength=0.9ex
                        \begin{picture}(2,2)
                        \linethickness{0.06ex}
                        \put(0.8,0){\line(4,5){0.8}}
                        \put(0.8,0){\line(-4,5){0.8}}
                        \put(0.8,2){\line(-4,-5){0.8}}
                        \put(0.8,2){\line(4,-5){0.8}}
                        \put(0.8,0.6){\line(0,1){0.8}}
                        \put(0.4,1){\line(1,0){0.8}}
                        \end{picture}}}
                        \,}

\newcommand{\sometimeb}{\,\raisebox{-.2ex}{
                        \mbox{\unitlength=0.9ex
                        \begin{picture}(2,2)
                        \linethickness{0.06ex}
                        \put(0.8,0){\line(4,5){0.8}}
                        \put(0.8,0){\line(-4,5){0.8}}
                        \put(0.8,2){\line(-4,-5){0.8}}
                        \put(0.8,2){\line(4,-5){0.8}}
                        \put(0.4,1){\line(1,0){0.8}}
                        \end{picture}}}
                        \,}

\newcommand{\sometimefb}{\,\raisebox{-.2ex}{
                        \mbox{\unitlength=0.9ex
                        \begin{picture}(2,2)
                        \linethickness{0.06ex}
                        \put(0.8,0){\line(4,5){0.8}}
                        \put(0.8,0){\line(-4,5){0.8}}
                        \put(0.8,2){\line(-4,-5){0.8}}
                        \put(0.8,2){\line(4,-5){0.8}}
                        \put(0.4,0.7){\line(1,0){0.8}}
                        \put(0.4,1.3){\line(1,0){0.8}}
                        \put(0.8,0.9){\line(0,1){0.8}}
                        \end{picture}}}
                        \,}


 \newcommand{\unintf}{\raisebox{-.2ex}{
                           \mbox{\unitlength=0.9ex
                           \begin{picture}(2,2)
                           \linethickness{0.06ex}
                           \put(0,0){\line(1,0){2}}
                           \put(0,2){\line(1,0){2}}
                           \put(0,0){\line(0,1){2}}
                           \put(2,0){\line(0,1){2}}
                           \put(1,0.6){\line(0,1){0.8}}
                           \put(0.6,1){\line(1,0){0.8}}
                           \put(1,2.3){\makebox(0,0){$\sim$}}
                           \end{picture}}}
                      \,}

 \newcommand{\unintb}{\raisebox{-.2ex}{
                           \mbox{\unitlength=0.9ex
                           \begin{picture}(2,2)
                           \linethickness{0.06ex}
                           \put(0,0){\line(1,0){2}}
                           \put(0,2){\line(1,0){2}}
                           \put(0,0){\line(0,1){2}}
                           \put(2,0){\line(0,1){2}}
                           \put(0.6,1){\line(1,0){0.8}}
                           \put(1,2.3){\makebox(0,0){$\sim$}}
                           \end{picture}}}
                      \,}


%
% 1b: Some less common temporal operators:
%     -----------------------------------
%
% We also have some variations on the ``\until'' operator, i.e., weak,
% Kr\"oger weak, Kr\"oger strong etc.

\def\skuntil{\,{\cal U}_k\,}
\def\wuntil{\,{\cal U}_w\,}
\def\wkuntil{\,{\cal U}_{kw}\,}

% Now for some TLR operators:
%
\def\untilp{\,{\cal U}^{+}\,}
\def\unlessp{\unless^+}

% We also have Kr\"ogers `atnext' operator (in text form until a
% better version is invented).
\def\atnext{\,\hbox{\bf atnext}\,}

% and the good old EL next action operator:
% (again like this until a better symbol is found)
\def\Act{\hbox{\rm\bf A} }
\def\BAct{\hbox{\rm\bf B} }
\def\CAct{\hbox{\rm\bf C} }
\def\Actv{\Act_{\bar{v}}}
\def\Acty{\Act_{\bar{y}}}
\def\Actvuy{\Act_{\bar{v}\cup\bar{y}}}
\def\Actz{\Act_{\bar{z}}}
\def\Actx{\Act_{\bar{x}}}
\def\BActv{\BAct_{\bar{v}}}
\def\Actpr{\Act_{\hbox{\rm PROP}}}

%
% 1c: Some fixpoint-type operators:
%     ----------------------------
%
% The standard fixpoint operators:
\def\lstfix#1#2{\mu#1.\, #2{#1}}
\def\grfix#1#2{\nu#1.\, #2{#1}}
\def\lstfixapp#1#2{\mu#1.\, #2 (#1 )}
\def\grfixapp#1#2{\nu#1.\, #2 (#1 )}

% We can also define some common fixpoints as follows:
% always
\def\alwfix#1#2{\nu#2 .(#1\land\next#2)}

% alwayvs, but using next action instead of next:
\def\actionalwfix#1#2{\nu#2 .(#1\land\Act#2)}

% Various sometime, and until versions. Strong, weak, using next, or
% next action.
\def\weaksomefix#1#2{\mu#2 .(#1\,\lor\next#2)}
\def\weakuntfix#1#2#3{\mu#3 .(#2\,\lor (#1\land\next#3))}
\def\unlessfix#1#2#3{\nu#3 .(#2\,\lor (#1\land\next#3))}
\def\somefix#1#2{\mu#2 .(#1\lor\snext#2)}
\def\actionsomefix#1#2{\mu#2 .(#1\lor\Act#2)}
\def\untfix#1#2#3{\mu#3 .(#2\,\lor (#1\land\snext#3))}
\def\actionuntfix#1#2#3{\mu#3 .(#2\,\lor (#1\land\Act#3))}
\def\prodtheta#1{\theta (#1)}
\def\prodchi#1{\chi (#1)}
\def\prodGamma#1{\Gamma (#1)}
\def\negprodnegchi#1{\neg\chi(\neg#1)}
\def\negprodnegtheta#1{\neg\theta(\neg#1)}
\def\negprodnegGamma#1{\neg\Gamma(\neg#1)}

%
% 1d: Some CTL and associated operators:
%     ---------------------------------
\def\ctlFinf{\!\!\!\!\hbox{\renewcommand{\arraystretch}{0.1} %
             \begin{tabular}[b]{c} $\infty$\\ F\end{tabular}}\!\!\!{}}
\def\ctlGalm{\hbox{\renewcommand{\arraystretch}{0.1} %
             \begin{tabular}[b]{c} $\infty$\\ G\end{tabular}}}
\def\infoften{\hbox{\renewcommand{\arraystretch}{0.1} %
             \begin{tabular}[b]{c} $\infty$\\ $\sometime$\end{tabular}}}
\def\infR{\renewcommand{\arraystretch}{0.1} %
             \begin{array}[b]{c} \infty\\ \sometime\end{array}}
\def\almalways{\hbox{\renewcommand{\arraystretch}{0.4} %
             \begin{tabular}[b]{c} $\infty$\\ ${\!\raisebox{-.3ex}{
                           \mbox{\unitlength=0.9ex
                           \begin{picture}(2,2)
                           \linethickness{0.06ex}
                           \put(0,0){\line(1,0){2}}
                           \put(0,2){\line(1,0){2}}
                           \put(0,0){\line(0,1){2}}
                           \put(2,0){\line(0,1){2}}
                           \end{picture}}}
                      \,}$\end{tabular}}}
\def\almR{\renewcommand{\arraystretch}{0.4} %
             \begin{array}[b]{c} \infty\\ {\!\raisebox{-.3ex}{
                           \mbox{\unitlength=0.9ex
                           \begin{picture}(2,2)
                           \linethickness{0.06ex}
                           \put(0,0){\line(1,0){2}}
                           \put(0,2){\line(1,0){2}}
                           \put(0,0){\line(0,1){2}}
                           \put(2,0){\line(0,1){2}}
                           \end{picture}}}
                      \,}\end{array}}
%
% Section 2: Some other standard mathematical symbols
% ===================================================
%
% 2a: Standard domains
%     ----------------
%
% Now for the standard Natural, Real, and Integer symbols: %
%\font\symbols=msym10
%\def\Nat{\hbox{\symbols\char"4E}}
%\def\Int{\hbox{\symbols\char"5A}}
%\def\Real{\hbox{\symbols\char"52}}
%\def\Bool{\hbox{\rm\bf B}}
%\def\Ratl{\hbox{\rm\bf Q}}
% Unfortunately, some idiot didn't provide gf files for msym10, so we
% have to resort to the old bold rubbish again:
\def\Nat{\hbox{\rm \bf N}}
\def\Int{\hbox{\rm \bf Z}}
\def\Real{\hbox{\rm \bf R}}
\def\Bool{\hbox{\rm\bf B}}
\def\Ratl{\hbox{\rm\bf Q}}

% 2b: Standard logical operations
%     ---------------------------
%
% and a few logical aliases you might expect

\let\A=\forall
\let\E=\exists
\def\Eunique{\exists !}
\def\ax{\vdash\quad}

% Some more standard logical operators:
\let\imp=\rightarrow
\def\iffe{\,\Leftrightarrow\,}
\let\iff=\Leftrightarrow
\let\prv=\vdash
\def\lland{\,\land\,}
\def\bland{\ \land\ }
\def\llor{\,\lor\,}

% now for a few more sensible definitions, namely macros for
% IFF, NOT and AND, where these roman font words appear in a
% definition (again to be used in math font only)

\def\IFF{\quad\hbox{\rm iff}\quad}
\def\IF{\ \hbox{\rm if}\ }
\def\THEN{\ \hbox{\rm then}\,}
\def\ELSE{\ \hbox{\rm else}\,}
\def\NOT{\ \hbox{\rm not}\,}
\def\AND{\ \hbox{\rm and}\  }
\def\TAND{\ \hbox{\rm and}\ }
\def\OR{\ \hbox{\rm or}\ }
\def\SuTh{\ \hbox{\rm such that}\ }
\def\PROP{\hbox{\rm PROP}\,}
\def\PROG{\hbox{\rm PROG}\,}
\def\ASS{\hbox{\rm ASS}\,}

% Some funny (but dangerous) versions of logical values:

% Here's a useful macro for changing fonts in the new latex
\def\normal{\ifx\selectfont\undefined\else\normalshape\fi}

\def\ltrue{\hbox{\rm\bf true}}
\def\lfalse{\hbox{\rm\bf false}}
\def\true{\hbox{\bf True}}
\def\false{\hbox{\bf False}}

% and a few from set theory:
\let\union=\cup
\let\Union=\bigcup
\let\intersection=\cap
\let\Intersection=\bigcap
\let\intersect=\cap
\let\Intersect=\bigcap

% Section 3: Various other symbols (in subsections)
% =================================================
%
% 3a: CCS, and CSP symbols.
%     --------------------
%
% And now some of Graham's CCS/CSP macros:
\newcommand{\fatbar}{\raisebox{-.25ex}{
                           \mbox{\unitlength=1ex
                           \begin{picture}(1,2)
                           \linethickness{0.05ex}
                           \put(0,0){\line(1,0){1}}
                           \put(0,2){\line(1,0){1}}
                           \put(0,0){\line(0,1){2}}
                           \put(1,0){\line(0,1){2}}
                           \end{picture}}}
                      \;}
\newcommand{\openfatbar}{\raisebox{-.25ex}{
                           \mbox{\unitlength=1ex
                           \begin{picture}(1,2)
                           \linethickness{0.05ex}
                           %\put(0,0){\line(1,0){1}}
                           \put(0,2){\line(1,0){1}}
                           \put(0,0){\line(0,1){2}}
                           \put(1,0){\line(0,1){2}}
                           \end{picture}}}
                      \;}
\newcommand{\twolines}{\raisebox{-.25ex}{
                           \mbox{\unitlength=1ex
                           \begin{picture}(1,2)
                           \linethickness{0.05ex}
                           %\put(0,0){\line(1,0){1}}
                           %\put(0,2){\line(1,0){1}}
                           \put(0,0){\line(0,1){2}}
                           \put(0.5,0){\line(0,1){2}}
                           \end{picture}}}
                      \;}
\newcommand{\twolinessub}{\raisebox{-.25ex}{
                           \mbox{\unitlength=1ex
                           \begin{picture}(1,2)
                           \linethickness{0.05ex}
                           %\put(0,0){\line(1,0){1}}
                           %\put(0,2){\line(1,0){1}}
                           \put(0,0){\line(0,1){2}}
                           \put(0.5,0){\line(0,1){2}}
                           \end{picture}}}}

\newcommand{\threelines}{\raisebox{-.25ex}{
                           \mbox{\unitlength=1ex
                           \begin{picture}(1,2)
                           \linethickness{0.05ex}
                           %\put(0,0){\line(1,0){1}}
                           %\put(0,2){\line(1,0){1}}
                           \put(0,0){\line(0,1){2}}
                           \put(0.5,0){\line(0,1){2}}
                           \put(1,0){\line(0,1){2}}
                           \end{picture}}}
                      \;}
\newcommand{\pref}{\rightarrow}
\newcommand{\dor}{\: \fatbar \:}
\newcommand{\choice}{\fatbar }
\newcommand{\parcomp}{\|}
\newcommand{\ndor}{\: \openfatbar \:}
%\newcommand{\csppar}{\: \parallel \:}
%\newcommand{\cspint}{\: \mid\mid\mid \:}
\newcommand{\csppar}{\: \twolines \:}
\newcommand{\cspint}{\: \threelines \:}

%
% 3b: Dynamic and Process Logic Symbols
%     ---------------------------------

% As the \[lr]angle symbols are too thin, and <> are too wide for
% dynamic logic <a>p brackets, we've defined these:
%
%\newcommand{\DLlangle}{\raisebox{-.25ex}{
%                          \mbox{\unitlength=1ex
%                          \begin{picture}(1,2)
%                          \linethickness{0.05ex}
%                          \put(0,1){\line(1,1){1}}
%                          \put(0,1){\line(1,-1){1}}
%                          \end{picture}}}}
%
%\newcommand{\DLrangle}{\raisebox{-.25ex}{
%                          \mbox{\unitlength=1ex
%                          \begin{picture}(1,2)
%                          \linethickness{0.05ex}
%                          \put(0,0){\line(1,1){1}}
%                          \put(0,2){\line(1,-1){1}}
%                          \end{picture}}}}
%
% Yes, you guessed it, these don't work! Apparently the minimum length
% of angled line you can have is 2ex, which is a bit too long!
% So, we'll resort to the following for the time being:
\let\DLlangle=\langle
\let\DLrangle=\rangle

% For Process Logics:
\def\f{\,\hbox{\rm f}\,}
\def\fX{\,\hbox{\rm f}\, X}
\def\suf{\/\hbox{\rm suf}\,}
\def\first{\,\hbox{\rm first}}
\def\firstu{\,\hbox{\rm first(}u\hbox{\rm )}\,}
\let\fusion=\circ

% and Dynamic Logics:
\def\Ra{R_\alpha}
\def\Rb{R_\beta}
\def\bab{\DLlangle\alpha\DLrangle}
\def\babX{\DLlangle\alpha\DLrangle X}
\def\asomep{\lbrack\alpha\rbrack\,\hbox{\rm some}\, p}
\def\astar{{\alpha^{\ast}}}
\def\asupi{{\alpha^i}}

% and some useful abbreviations (again pretty lazy):
\def\RR{\hbox{$\cal R$} }
\def\RRsubR{\, {\cal R}_R \, }
\def\DD{\hbox{$\cal D$} }
\def\DDQ{``${\cal D}$''}
\def\OOQ{``${\cal O}$''}
\def\SSQ{``${\cal S}$''}
\def\LLQ{``${\cal L}$''}
\def\CCQ{``${\cal C}$''}
\def\SS{\hbox{$\cal S$} }
\def\MM{\hbox{$\cal M$} }
\def\NN{\hbox{$\cal N$}{}}
\def\NNsubE{\, {\cal N}_E \, }
\def\K{\hbox{\rm\bf K} }
\def\T{\hbox{\rm\bf T} }
\def\cL{\hbox{${\cal L}$} }

% +++++++++++++++++++++++++++++++++++++++++++++++++++++++++++++++++++++

% A little help for those of us who define semantics fairly often:
%
% As the symbols `<A,s_0> |= ' are used so much in semantic definitions
% we have the following macro:
% N.B., all to be used in math mode only.

% The new standard model macro is:
\def\MOD#1#2{\,\langle{\cal #1},\, #2\rangle\,\models\ }

\makeatother

% The `vdm-like' definition symbol:
% ( a triangle over = )
%
%\def\DEFINE{\raise.5ex  %
%       \hbox{\footnotesize\underline{$\mathchar"3\cmsy@34$}}}
%
\def\DEFINE{\,\stackrel{\hbox{{\rm\tiny def}}}{=}\, }

