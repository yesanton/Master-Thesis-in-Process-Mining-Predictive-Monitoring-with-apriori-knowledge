Predictive business process monitoring aims at leveraging past process execution data to predict how ongoing (uncompleted) process executions will unfold up to
their completion.
%State-of-the-art techniques rely on the data of the
%past in order to be able to provide predictions on the future of a
%currently running case.
Nevertheless, cases exist in which, together with past
execution data, some additional knowledge (a-priori
knowledge) about how a process execution will develop in the future is available. This knowledge about the future can be leveraged for
improving the quality of the predictions of events that are currently unknown. In this thesis, we present two
techniques - based on Recurrent Neural Networks with Long Short-Term Memory (LSTM) cells - able to
leverage knowledge about the structure of the process execution traces as well as a-priori knowledge about how they will unfold in the future for predicting the sequence of future activities of ongoing process executions. The results obtained by applying these techniques on six real-life logs show an improvement in terms of accuracy over a plain
LSTM-based baseline.
%Predictive Business Process monitoring aims at leveraging past event logs to predict how ongoing (uncompleted) cases will unfold up to their completion. State-of-the-art techniques mainly rely on the data of the past in order to be able to provide predictions on the future of the current trace. Nevertheless, cases exist in which, together with past execution data, some apriori knowledge on the future is available. In this paper we present a family of techniques based on Recurrent Neural Networks which are able to leverage knowledge obtained from the structure of the logs as well as apriori knowledge for the prediction of the sequence of future activities in ongoing traces. Results on real life logs show that the proposed techniques are able to improve the accuracy of state-of-the-art results.
%%
%Predictive Business Process monitoring methods use completed logs in order to build predictive systems that work at runtime. Most of the methods thereof concerned with prediction of the following sequence of activities, the remaining time, or compliance with a measure.  Often, in real world applications there is some information about the future of the running process is available. Existing methods for prediction of the continuation of ongoing trace suffer of low accuracy, and performance inconsistency. In this thesis the problems of existing state of the art methods are explored. Also, the methods based on RNN Neural Networks with LSTM and Linear Temporal Logic formulas that account for knowledge given on online prediction are proposed. Finally, they are evaluated on real life business logs and the benefits are discussed.

