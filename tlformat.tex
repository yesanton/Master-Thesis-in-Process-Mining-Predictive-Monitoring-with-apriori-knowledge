% The mdf sponsored, miw designed macro to put text back in
% roman in while in math mode:
%
%Here are some quotes:" % prints quotes
%
%And in math mode, $x="some roman text"$.
{\catcode`\"=\active
\gdef\turnonquotes{\catcode`\"=\active \let"=\dobox}
\gdef\dobox#1"{\hbox{\rm #1}}}

\everymath={\turnonquotes}
\everydisplay={\turnonquotes}


% The signature of functions can be done using the following 2 macros
%
\def\funcsig#1#2{#1\rightarrow{#2}}
\def\Signature#1#2{#1\colon\, #2}
\let\Sig=\Signature

% Put a box around some text:
\def\parlinebox#1{\begin{center}
                  \fbox{\parbox{4.5in}{#1}}
                  \end{center}}

% To put a silly quote at the beginning of your text, do:
% \sillyquote{Some text}{By whom} right at the beginning of your
% chapter.
\def\sillyquote#1#2{%
        \begin{flushright}
        \parbox[t]{3in}{{\noindent #1}
                
        \def\baselinestretch{1}
        {\it #2}}
        \end{flushright}
        \vspace{1cm}}

\def\vsillyquote#1#2{%
        \begin{flushright}
        \parbox[t]{3in}{{\noindent #1}
                
        \def\baselinestretch{1}
        \begin{flushright}{\it #2}\end{flushright}}
        \end{flushright}
        \vspace{1cm}}

%To put an old-fashioned centerline across the page:
\def\mycentreline{\begin{center} \rule{3.8in}{0.005in}\end{center}}

% Now on to `proper' semantics!
% We've found that sometimes (e.g., using crappy printers that blur our 
% symbols) the obvious use of \! in [\! [ etc moves the 
% brackets a little too close together, so we have:

\def\squash{\mskip-2.5mu}
% and we can replace all obvious occurences of \! by \squash ...
% That's if you like it that way.
% (obviously this can be changed easily back to 3 or 2.5 etc.,..

\def\leftsem{\lbrack\!\lbrack}
\def\rightsem{\rbrack\!\rbrack}
% these two definitions are used consistently, so if a better way 
% is found of doing these brackets (e.g., pictures) just substitute
% in these two and the rest will follow

\def\SEMt#1{\hbox{$\leftsem #1\rightsem $}}
\def\SEM#1{\leftsem #1\rightsem}
\def\DENSEM#1{{\cal D}\leftsem #1\rightsem}
\def\OPSEM#1{{\cal O}\leftsem #1\rightsem}
\def\SSSEM#1{{\cal S}\leftsem #1\rightsem}
\def\GENSEM#1#2{{\cal #1}\leftsem #2\rightsem}
\def\TSEM#1{{\cal T}\leftsem #1\rightsem}

% Now for some macros useful for talks on temporal semantics!
% First, one of Grahams
\def\Vlongline{\hbox to 140pt{\hrulefill}}
\newcommand{\infrule}[2]{{\renewcommand{\arraystretch}{0.6}\begin{array}{c} #1 \\       \Vlongline \\#2 \\ \end{array}}}
% and now for michael's hack of grahams beautiful makerow ...
\newcommand{\superinfrule}[3]{{\renewcommand{\arraystretch}{0.6}\begin{array}{#1} #2 \\ \Vlongline \\#3 \\ \end{array}}}

% and for a more general version of both these:
% The following macro takes 4 arguments. These are interpreted as follows:
% #1 --- defines the positioning of the array. (N.B., if other than l,
%        c, or r are used, `&'s must be in the arguments somewhere
% #2 --- argument for the top element of the inference rule
% #3 ---   "       "   "  bottom element.
% #4 --- optional arraystretch parameter
%
\newcommand{\inferencerule}[4]{\renewcommand{\arraystretch}{#4}\begin{array}{#1}{}#2\\\hline{}#3\end{array}}

% Now one of Graham's environments to make comments stand out:
%
% \newenvironment{comment}%
%         {\begin{center}\begin{minipage}%
%                         {.9\textwidth}\small\tt}%
%                       {\end{minipage}%
%          \end{center}}

% Now a couple of environments for doing arrays in square brackets:
%
\newenvironment{sqarray}[2]{\left[ \begin{array}[#1]{#2}}{\end{array} \right]}
                        
